\documentclass{article}
\usepackage{graphicx} % Required for inserting images
\usepackage{xeCJK}
\usepackage{cite} % 推荐加上

\title{DFA-SAGE}
\author{Desong Lin}
\date{September 2025}

\begin{document}

\maketitle

\section{Introduction}

物联网(IoT)已广泛部署于工业、医疗保健和智能家居等各类领域,为用户提供了无与伦比的便利。然而,随着物联网设备数量的持续增长,针对物联网网络的攻击也变得愈发频繁和复杂 \cite{7123563}。网络攻击者利用物联网设备固有的资源限制和安全漏洞执行恶意活动,例如分布式拒绝服务(DDoS)攻击、网络欺骗和恶意软件传播,这对用户隐私和设备可靠运行构成了重大风险,从而对物联网系统的安全性带来了显著挑战。

% \cite{electronics9101565}

网络入侵检测系统(Network Intrusion Detection Systems, NIDS)是防御物联网网络攻击的关键技术,它能够及时检测设备间的异常通信和未授权访问。基于特征的入侵检测系统(Signature-based IDS)通过匹配已知攻击特征实现较低的误报率,但难以检测新型攻击 \cite{electronics9101565}。相比之下,基于行为的系统能够识别未知攻击,其中机器学习技术通过构建模型分析复杂、多维数据以实现异常检测 \cite{Hazman2023} \cite{Al-Ambusaidi2024}。尤其是深度学习方法,在自动特征提取方面表现出色,并在处理高维复杂数据方面具有卓越性能 \cite{9796521} \cite{Hanafi2024} \cite{Saravanan2024}。然而,现有基于机器学习和深度学习的 NIDS 方法通常依赖于平面数据结构提取流量特征,这限制了其捕捉物联网数据中固有空间拓扑信息的能力 \cite{10258187},从而降低了在物联网环境中的检测效果。

随着图神经网络(Graph Neural Networks, GNNs)的出现,越来越多的研究将其引入入侵检测领域 \cite{ZHONG2024103821}。GNN 主要通过信息聚合和节点更新学习节点和图的特征表示,从而能够有效建模复杂的网络结构,并捕捉图中节点间的关系。这使得 GNN 在物联网环境中具有明显优势 \cite{sanchez-lengeling2021a}。例如,GCN-Ensemble 融合模型 \cite{Mittal2024} 将流量数据转化为图结构,并整合卷积神经网络集成进行分类。这一方法有效解决了物联网入侵检测中数据不平衡和拓扑信息不足的问题,在多个数据集上取得了优异性能。在基于 GNN 的研究中,常用的图构建方法是将流量数据中的源地址和目标地址映射为图节点,而将网络流特征映射为边。然而,该方法将 IP 地址中包含的空间拓扑信息转化为图的拓扑结构,使节点本身缺乏内在特征。一些研究通过将节点特征初始化为元素全为 1 的向量,但这些特征仍缺乏区分性。此外,尽管 GNN 擅长捕捉流量数据中的复杂关系,许多方法仍主要关注节点特征的聚合与表示学习,或仅在图构建和基本权重调整阶段使用边特征。这种方法忽视了边特征在反映流量模式和捕捉攻击行为中的潜力,未能充分发挥其在图结构学习中的作用。

为了应对物联网(IoT)入侵检测的挑战,本文提出了用于物联网入侵检测的 Dual-Fusion Aggregation SAGE (DFA-SAGE) 模型。

在图构建阶段,我们引入了一种基于邻近边特征聚合的节点初始化方法,利用邻接边的信息为每个节点提供独特的特征,从而增强其表达能力,克服传统方法中节点特征无效的问题。

在消息聚合过程中,我们采用双步融合聚合机制(Dual-Fusion Aggregation):
第一步(边特征强化):执行一次基于边特征的消息传递,用于强化局部边信息并初始化下一轮节点特征。

第二步(节点-边混合融合):执行第二次聚合,同时聚合上一轮更新的节点特征和原始边特征,并采用独特的自强化节点更新函数(例如,对聚合后的邻居特征进行相加操作)来深度融合特征表示。
通过独特的双步聚合操作,模型能够在单层 GNN 中有效捕获二跳邻域的边特征,从而强化局部结构信息,同时显著降低多层堆叠带来的计算复杂度和过度平滑风险。此外,在边嵌入阶段,模型保留原始边特征并与节点特征结合,采用特征缩放和相加的非线性预测,以进一步增强边特征在捕捉流量模式和攻击行为中的潜力。

总之,本文主要有三方面贡献:

1. 提出基于邻接边特征聚合的节点初始化方法: 我们提出了一种基于邻接边特征聚合的节点初始化方法,通过整合邻接边的信息赋予每个节点独特特征,从而增强其表达能力,并解决传统图构建方法中节点特征无效的问题。

2. 设计高效的双步融合聚合机制 (Dual-Fusion Aggregation): 在消息聚合阶段,我们设计了 DFA-SAGE 的双步聚合机制:第一步强化边特征,第二步融合边和节点特征,并采用独特的自强化节点更新规则。这种机制使得通过单次 DFA-SAGE 层的聚合操作即可有效获取二跳邻域的边特征,同时缓解了多层 GNN 带来的过度平滑问题,保持了模型的计算效率。

3. 验证模型的高性能和潜力: DFA-SAGE 模型在 CICIoT2023、Edge-IIoT 和 BoT-IoT 数据集上表现出色,准确率和 F1 值始终高于 98.6%,显著优于现有方法,体现了其在物联网入侵检测中的巨大潜力。

本文结构如下:第 2 节回顾了利用图神经网络进行物联网入侵检测的相关工作;第 3 节详细介绍了数据预处理步骤,并全面说明了所提出的 EE-GraphSAGE 方法;第 4 节展示实验设置并分析实验结果;第 5 节总结研究并归纳关键发现。


\bibliographystyle{IEEEtran}   % 或 plain, unsrt, alpha...
\bibliography{ref}             % ref.bib

\end{document}
